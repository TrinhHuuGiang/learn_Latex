%                            ================================
%                            [Type doctument]
%                            - \documentclass[option]{class}
%                            [Other page setup]
%                            - geometry: set margin, should
%                                similar with \documentclass
%                            ================================
\documentclass[12pt, a4paper]{report}
\usepackage[a4paper,inner=3cm,outer=2cm]{geometry}

%============================================================
% Preamble
% - Include packages allow expand ability of Texlive by more 
%       commands, environments
% - Commands effect entire document
%============================================================
%                            ================================
%                            Package
%                            - lipsum: lorem paragraph
%                            Url
%                            - hyperref: url
%                            Color
%                            - color
%                            Language format
%                            - inputenc, fontenc, babel
%                            - lmodern: Latin modern
%                            ================================
\usepackage{lipsum}
\usepackage[hidelinks]{hyperref}
\usepackage{color}


\usepackage[utf8]{inputenc}
\usepackage[T5,T1]{fontenc}
\usepackage{lmodern} % a Latin font version enhance from
                        % cm-supper + vector format + better support T1,T5
\usepackage[vietnamese, english]{babel} 

%============================================================
% Document
% - Plain text, list of tables, figures, input other tex, 
%       bibilography, ...
% - Commands with scope: in group
% - Environments differ from 'document': 
%       /begin{env} , /end{env}
%============================================================
\begin{document}
%                                                        ================================
%                                                        [Front matter]
%                                                        2. Empty
%                                                        3. Title page
%                                                        4. Information (copyright notice, ISBN, etc.)
%                                                        5. Dedication if any, else empty
%                                                        6. Table of contents
%                                                        7. List of figures (can be in the backmatter too)
%                                                        8. Preface chapter
%                                                        ================================


%                            ================================
%                            Title
%                            ================================
\title{Our Fun Document}
\author{Boby \and Alice}
\date{\today}
\maketitle

%                            ================================
%                            Abstract
%                            ================================
\renewcommand{\abstractname}{New abstract \LaTeX{} title}
\begin{abstract}

    This is abstract
    \lipsum[1]

\end{abstract}


%                            ================================
%                            Table of contents
%                            - TOC, LOF, LOT
%                            ================================
\renewcommand{\contentsname}{** Table of Contents **}
\tableofcontents

\renewcommand{\listtablename}{** List of Figures **}
\renewcommand{\listfigurename}{** List of Tables **}
\listoffigures
\listoftables

%                                                        ================================
%                                                        [Main matter]
%                                                        - Main topics
%                                                        ================================
\part{This is a part title}
\chapter{This is a chapter title}
\section[This is Short section title]{This is a section title}
\subsection{This is a sub section title}
\subsubsection{This is a sub sub section title}
\paragraph{This is a paragraph}
\subparagraph{This is a sub paragraph}


% linespread
\newpage

xxxxxxxxxxxxxxxxxxxxxxxxxxxxxx\\
\linespread{1}\selectfont
start normal linespread, factor 1\\
\lipsum[1]\\


xxxxxxxxxxxxxxxxxxxxxxxxxxxxxx\\
\linespread{0.4}\selectfont
start linespread factor 0.4\\
\lipsum[2]\\


xxxxxxxxxxxxxxxxxxxxxxxxxxxxxx\\
\linespread{1.6}\selectfont
start linespread factor 1.6\\
\lipsum[3]\\


% auto fill by: vfill hfill
\linespread{1}\selectfont
\newpage

\lipsum[1]

\vfill

here left \hfill Here central \hfill Here right\\

\vspace{-2cm} Row with vspace -2cm\\

\hspace{-2cm} Row with hspace -2cm\\

\vfill

\lipsum[2]


% no breaking sentence: fbox hbox
\newpage 

\fbox{\lipsum[1]}\\
\hbox{\lipsum[2]}\\

% url
Here is example url: \url{https://www.example-giangtrinh-1234.test/path/to/resources}\\

% verb (verbatim on 1 line)
Here is line verbatim: \verb=~!@#$%^&*()_=\\

Here is another verb : \verb**hello(^_^)*\\ 


% paragraph alignment
\newpage

\begin{flushleft}
    Below is left justified:\\
    \lipsum[1]\\
\end{flushleft}

\begin{flushright}
    Below is right justified:\\
    \lipsum[2]\\
\end{flushright}

\begin{center}
    Below is centering:\\
    \lipsum[3]\\
\end{center}

\vspace{1cm}Paragraph test done, return main page\\
\lipsum[4]\\


% verbatim  environment

\begin{verbatim}
hello `verbatim'

#include <stdio.h>
#include <stdint.h>

int main()
{
    fprintf(stderr,"Verbatim environment :v\n");
    return 0;
}

bye `verbatim'
\end{verbatim}



% color test
% "black", "white", "gray" x , "silver" x, 
% "maroon" x, "red", "purple" x, 
% "fuchsia" x, "green", "lime" x, "olive" x, 
% "yellow", "navy" x, "blue", "teal" x, "aqua" x
% => 6 color are accepted: black, white, red, green, yellow, blue
\newpage

This is color band for `color' package
\textcolor{black}{black} 
\textcolor{white}{white} white
% \textcolor{gray}{gray} 
% \textcolor{silver}{silver}
% \textcolor{maroon}{maroon}
\textcolor{red}{red} 
% \textcolor{purple}{purple} 
% \textcolor{fuchsia}{fuchsia}
\textcolor{green}{green} 
% \textcolor{lime}{lime}
% \textcolor{olive}{olive} 
\textcolor{yellow}{yellow}
% \textcolor{navy}{navy} 
\textcolor{blue}{blue}
% \textcolor{teal}{teal} 
% \textcolor{aqua}{aqua}

This is test color with wide space

\color{red}
\lipsum[1]\\
\color{black}


% color box: colorbox, fcolorbox
This is \colorbox{green}{\lipsum[1]} and below is fcolorbox\\
This is \fcolorbox{red}{yellow}{\lipsum[2]}\\


% test vietnames
\newpage

\selectlanguage{vietnamese}
\fontencoding{T5}
\selectfont

\textcolor{blue}
{
Người Việt Nam có thói quen tổ chức ăn uống chung.
Vì vậy các thành viên trong bữa ăn có quan hệ mật thiết và phụ thuộc nhau.
Điều này khác với phương Tây, vì mọi người đều có bữa ăn của riêng mình,
mọi người hoàn toàn độc lập với nhau. Họ ngược lại nên thích nói chuyện trong bữa ăn, 
trái ngược với người phương Tây tránh nói chuyện khi dùng bữa. 
Người Việt Nam không ăn quá nhanh hoặc quá chậm. Người Việt Nam không ăn quá nhiều hoặc quá ít.
Họ dùng đũa khi ăn và dùng cơm là một trong những lương thực chính. 
Các kiểu nấu ăn phổ biến nhất là luộc, xào, hấp hoặc hầm ...
Món ăn Việt Nam có xu hướng không nhiều dầu mỡ vì họ sử dụng tối thiểu dầu cho các món ăn của mình. 
Bữa ăn thông thường nhất là sự kết hợp của các loại thịt, cá và rau, rau thơm. 
Hầu hết các bữa ăn đều có thêm nước mắm hoặc nước tương.
Thức ăn thường được đặt ở giữa bàn. Người ít tuổi phải mời người lớn tuổi trước khi ăn.
Xem thêm tại: https://loigiaihay.com/bai-tap-310466.html
}




% 12 font style
\newpage
\selectlanguage{english}
\fontencoding{T1}
\selectfont

This is \textnormal{textnormal}\\ This is \textup{textup}\\
This is \textit{textit}\\ This is \textsl{textsl}\\
This is \textsc{textsc}\\ This is \uppercase{uppercase}\\
This is \textbf{textbf}\\ This is \textmd{textmd}\\
% This is \textlf{textlf}\\ % pdflatex no supper
This is \textrm{textrm}\\
This is \textsf{textsf}\\ This is \texttt{texttt}\\


% 10 font size
\vfill

{This is \tiny abCD tiny}\\
{This is \scriptsize abCD scriptsize}\\
{This is \footnotesize abCD footnotesize}\\
{This is \small abCD small}\\
{This is \normalsize abCD normalsize}\\
{This is \large abCD large}\\
{This is \Large abCD Large}\\
{This is \LARGE abCD LARGE}\\
{This is \huge abCD huge}\\
{This is \Huge abCD Huge}\\


hello



%                                                        ================================
%                                                        Appendix
%                                                        - Subordinate chapters
%                                                        ================================




%                                                        ================================
%                                                        Back matter
%                                                        - Bibliography
%                                                        - Glossary/Index
%                                                        ================================







%                            ================================
%                            END
%                            ================================
\end{document}

